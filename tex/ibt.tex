%==============================================================================
% tento soubor pouzijte jako zaklad
% this file should be used as a base for the thesis
% Autoři / Authors: 2008 Michal Bidlo, 2018 Jaroslav Dytrych
% Kontakt pro dotazy a připomínky: dytrych@fit.vutbr.cz
% Contact for questions and comments: dytrych@fit.vutbr.cz
%==============================================================================
% kodovani: UTF-8 (zmena prikazem iconv, recode nebo cstocs)
% encoding: UTF-8 (you can change it by command iconv, recode or cstocs)
%------------------------------------------------------------------------------
% zpracování / processing: make, make pdf, make clean
%==============================================================================
% Soubory, které je nutné upravit: / Files which have to be edited:
%   projekt-20-literatura-bibliography.bib - literatura / bibliography
%   projekt-01-kapitoly-chapters.tex - obsah práce / the thesis content
%   projekt-30-prilohy-appendices.tex - přílohy / appendices
%==============================================================================
\documentclass[]{fitthesis} % bez zadání - pro začátek práce, aby nebyl problém s překladem
%\documentclass[slovak]{fitthesis} % without assignment - for the work start to avoid compilation problem
%\documentclass[zadani]{fitthesis} % odevzdani do wisu a/nebo tisk s barevnými odkazy - odkazy jsou barevné
%\documentclass[slovak,zadani]{fitthesis} % for submission to the IS FIT and/or print with color links - links are color
%\documentclass[zadani,print]{fitthesis} % pro černobílý tisk - odkazy jsou černé
%\documentclass[slovak,zadani,print]{fitthesis} % for the black and white print - links are black
%\documentclass[zadani,cprint]{fitthesis} % pro barevný tisk - odkazy jsou černé, znak VUT barevný
%\documentclass[slovak,zadani,cprint]{fitthesis} % for the print - links are black, logo is color
% * Je-li práce psaná v anglickém jazyce, je zapotřebí u třídy použít 
%   parametr english následovně:
%   If thesis is written in english, it is necessary to use 
%   parameter english as follows:
%      \documentclass[english]{fitthesis}
% * Je-li práce psaná ve slovenském jazyce, je zapotřebí u třídy použít 
%   parametr slovak následovně:
%   If the work is written in the Slovak language, it is necessary 
%   to use parameter slovak as follows:
%      \documentclass[slovak]{fitthesis}
% * Je-li práce psaná v anglickém jazyce se slovenským abstraktem apod., 
%   je zapotřebí u třídy použít parametry english a enslovak následovně:
%   If the work is written in English with the Slovak abstract, etc., 
%   it is necessary to use parameters english and enslovak as follows:
%      \documentclass[english,enslovak]{fitthesis}

% Základní balíčky jsou dole v souboru šablony fitthesis.cls
% Basic packages are at the bottom of template file fitthesis.cls
% zde můžeme vložit vlastní balíčky / you can place own packages here

% Kompilace po částech (rychlejší, ale v náhledu nemusí být vše aktuální)
% Compilation piecewise (faster, but not all parts in preview will be up-to-date)
% \usepackage{subfiles}

% Nastavení cesty k obrázkům
% Setting of a path to the pictures
%\graphicspath{{obrazky-figures/}{./obrazky-figures/}}
%\graphicspath{{obrazky-figures/}{../obrazky-figures/}}

%---rm---------------
\renewcommand{\rmdefault}{lmr}%zavede Latin Modern Roman jako rm / set Latin Modern Roman as rm
%---sf---------------
\renewcommand{\sfdefault}{qhv}%zavede TeX Gyre Heros jako sf
%---tt------------
\renewcommand{\ttdefault}{lmtt}% zavede Latin Modern tt jako tt

% vypne funkci šablony, která automaticky nahrazuje uvozovky,
% aby nebyly prováděny nevhodné náhrady v popisech API apod.
% disables function of the template which replaces quotation marks
% to avoid unnecessary replacements in the API descriptions etc.
\csdoublequotesoff

% =======================================================================
% balíček "hyperref" vytváří klikací odkazy v pdf, pokud tedy použijeme pdflatex
% problém je, že balíček hyperref musí být uveden jako poslední, takže nemůže
% být v šabloně
% "hyperref" package create clickable links in pdf if you are using pdflatex.
% Problem is that this package have to be introduced as the last one so it 
% can not be placed in the template file.
\ifWis
\ifx\pdfoutput\undefined % nejedeme pod pdflatexem / we are not using pdflatex
\else
  \usepackage{color}
  \usepackage[unicode,colorlinks,hyperindex,plainpages=false,pdftex]{hyperref}
  \definecolor{hrcolor-ref}{RGB}{223,52,30}
  \definecolor{hrcolor-cite}{HTML}{2F8F00}
  \definecolor{hrcolor-urls}{HTML}{092EAB}
  \hypersetup{
	linkcolor=hrcolor-ref,
	citecolor=hrcolor-cite,
	filecolor=magenta,
	urlcolor=hrcolor-urls
  }
  \def\pdfBorderAttrs{/Border [0 0 0] }  % bez okrajů kolem odkazů / without margins around links
  \pdfcompresslevel=9
\fi
\else % pro tisk budou odkazy, na které se dá klikat, černé / for the print clickable links will be black
\ifx\pdfoutput\undefined % nejedeme pod pdflatexem / we are not using pdflatex
\else
  \usepackage{color}
  \usepackage[unicode,colorlinks,hyperindex,plainpages=false,pdftex,urlcolor=black,linkcolor=black,citecolor=black]{hyperref}
  \definecolor{links}{rgb}{0,0,0}
  \definecolor{anchors}{rgb}{0,0,0}
  \def\AnchorColor{anchors}
  \def\LinkColor{links}
  \def\pdfBorderAttrs{/Border [0 0 0] } % bez okrajů kolem odkazů / without margins around links
  \pdfcompresslevel=9
\fi
\fi
% Řešení problému, kdy klikací odkazy na obrázky vedou za obrázek
% This solves the problems with links which leads after the picture
\usepackage[all]{hypcap}

% Informace o práci/projektu / Information about the thesis
%---------------------------------------------------------------------------
\projectinfo{
  %Prace / Thesis
  project={BP},            %typ práce BP/SP/DP/DR  / thesis type (SP = term project)
  year={2018},             % rok odevzdání / year of submission
  date=\today,             % datum odevzdání / submission date
  %Nazev prace / thesis title
  title.cs={Název práce},  % název práce v češtině či slovenštině (dle zadání) / thesis title in czech language (according to assignment)
  title.en={Thesis title}, % název práce v angličtině / thesis title in english
  %title.length={14.5cm}, % nastavení délky bloku s titulkem pro úpravu zalomení řádku (lze definovat zde nebo níže) / setting the length of a block with a thesis title for adjusting a line break (can be defined here or below)
  %Autor / Author
  author.name={Jméno},   % jméno autora / author name
  author.surname={Příjmení},   % příjmení autora / author surname 
  %author.title.p={Bc.}, % titul před jménem (nepovinné) / title before the name (optional)
  %author.title.a={Ph.D.}, % titul za jménem (nepovinné) / title after the name (optional)
  %Ustav / Department
  department={UPGM}, % doplňte příslušnou zkratku dle ústavu na zadání: UPSY/UIFS/UITS/UPGM / fill in appropriate abbreviation of the department according to assignment: UPSY/UIFS/UITS/UPGM
  % Školitel / supervisor
  supervisor.name={Jméno},   % jméno školitele / supervisor name 
  supervisor.surname={Příjmení},   % příjmení školitele / supervisor surname
  supervisor.title.p={prof. RNDr.},   %titul před jménem (nepovinné) / title before the name (optional)
  supervisor.title.a={Ph.D.},    %titul za jménem (nepovinné) / title after the name (optional)
  % Klíčová slova / keywords
  keywords.cs={Sem budou zapsána jednotlivá klíčová slova v českém (slovenském) jazyce, oddělená čárkami.}, % klíčová slova v českém či slovenském jazyce / keywords in czech or slovak language
  keywords.en={Sem budou zapsána jednotlivá klíčová slova v anglickém jazyce, oddělená čárkami.}, % klíčová slova v anglickém jazyce / keywords in english
  %keywords.en={Here, individual keywords separated by commas will be written in English.},
  % Abstrakt / Abstract
  abstract.cs={Do tohoto odstavce bude zapsán výtah (abstrakt) práce v českém (slovenském) jazyce.}, % abstrakt v českém či slovenském jazyce / abstract in czech or slovak language
  abstract.en={Do tohoto odstavce bude zapsán výtah (abstrakt) práce v anglickém jazyce.}, % abstrakt v anglickém jazyce / abstract in english
  %abstract.en={An abstract of the work in English will be written in this paragraph.},
  % Prohlášení (u anglicky psané práce anglicky, u slovensky psané práce slovensky) / Declaration (for thesis in english should be in english)
  declaration={Prohlašuji, že jsem tuto bakalářskou práci vypracoval samostatně pod vedením pana X...
Další informace mi poskytli...
Uvedl jsem všechny literární prameny a publikace, ze kterých jsem čerpal.},
  %declaration={Hereby I declare that this bachelor's thesis was prepared as an original author’s work under the supervision of Mr. X
% The supplementary information was provided by Mr. Y
% All the relevant information sources, which were used during preparation of this thesis, are properly cited and included in the list of references.},
  % Poděkování (nepovinné, nejlépe v jazyce práce) / Acknowledgement (optional, ideally in the language of the thesis)
  acknowledgment={V této sekci je možno uvést poděkování vedoucímu práce a těm, kteří poskytli odbornou pomoc
(externí zadavatel, konzultant, apod.).},
  %acknowledgment={Here it is possible to express thanks to the supervisor and to the people which provided professional help
%(external submitter, consultant, etc.).},
  % Rozšířený abstrakt (cca 3 normostrany) - lze definovat zde nebo níže / Extended abstract (approximately 3 standard pages) - can be defined here or below
  %extendedabstract={Do tohoto odstavce bude zapsán rozšířený výtah (abstrakt) práce v českém (slovenském) jazyce.},
  %faculty={FIT}, % FIT/FEKT/FSI/FA/FCH/FP/FAST/FAVU/USI/DEF
  faculty.cs={Fakulta informačních technologií}, % Fakulta v češtině - pro využití této položky výše zvolte fakultu DEF / Faculty in Czech - for use of this entry select DEF above
  faculty.en={Faculty of Information Technology}, % Fakulta v angličtině - pro využití této položky výše zvolte fakultu DEF / Faculty in English - for use of this entry select DEF above
  department.cs={Ústav matematiky}, % Ústav v češtině - pro využití této položky výše zvolte ústav DEF nebo jej zakomentujte / Department in Czech - for use of this entry select DEF above or comment it out
  department.en={Institute of Mathematics} % Ústav v angličtině - pro využití této položky výše zvolte ústav DEF nebo jej zakomentujte / Department in English - for use of this entry select DEF above or comment it out
}

% Rozšířený abstrakt (cca 3 normostrany) - lze definovat zde nebo výše / Extended abstract (approximately 3 standard pages) - can be defined here or above
%\extendedabstract{Do tohoto odstavce bude zapsán výtah (abstrakt) práce v českém (slovenském) jazyce.}

% nastavení délky bloku s titulkem pro úpravu zalomení řádku - lze definovat zde nebo výše / setting the length of a block with a thesis title for adjusting a line break - can be defined here or above
%\titlelength{14.5cm}


% řeší první/poslední řádek odstavce na předchozí/následující stránce
% solves first/last row of the paragraph on the previous/next page
\clubpenalty=10000
\widowpenalty=10000

% checklist
\newlist{checklist}{itemize}{1}
\setlist[checklist]{label=$\square$}

\begin{document}
  % Vysazeni titulnich stran / Typesetting of the title pages
  % ----------------------------------------------
  \maketitle
  % Obsah
  % ----------------------------------------------
  \setlength{\parskip}{0pt}

  {\hypersetup{hidelinks}\tableofcontents}
  
  % Seznam obrazku a tabulek (pokud prace obsahuje velke mnozstvi obrazku, tak se to hodi)
  % List of figures and list of tables (if the thesis contains a lot of pictures, it is good)
  \ifczech
    \renewcommand\listfigurename{Seznam obrázků}
  \fi
  \ifslovak
    \renewcommand\listfigurename{Zoznam obrázkov}
  \fi
  % \listoffigures
  
  \ifczech
    \renewcommand\listtablename{Seznam tabulek}
  \fi
  \ifslovak
    \renewcommand\listtablename{Zoznam tabuliek}
  \fi
  % \listoftables 

  \ifODSAZ
    \setlength{\parskip}{0.5\bigskipamount}
  \else
    \setlength{\parskip}{0pt}
  \fi

  % vynechani stranky v oboustrannem rezimu
  % Skip the page in the two-sided mode
  \iftwoside
    \cleardoublepage
  \fi

  % Text prace / Thesis text
  % ----------------------------------------------
  %===============================================================================
% Autoři: Michal Bidlo, Bohuslav Křena, Jaroslav Dytrych, Petr Veigend a Adam Herout 2018
\chapter{Úvod}
\label{uvod}
TODO

\chapter {Dopplerov radar}
\label{doppler}
TODO

\chapter {Zpracovanie obrazu}
\label{camera}
TODO

\chapter{Fúzia radaru s obrazom}
\label{fusion}
TODO

\chapter{Analýza a návrh riešenia}
\label{analysis}
TODO

\chapter{Implementácia}
\label{implementation}
TODO

\chapter{Závěr}
\label{zaver}
TODO

\nocite{Chavez-Garcia}



%===============================================================================

  
  % Kompilace po částech (viz výše, nutno odkomentovat)
  % Compilation piecewise (see above, it is necessary to uncomment it)
  %\subfile{projekt-01-uvod-introduction}
  % ...
  %\subfile{chapters/projekt-05-conclusion}


  % Pouzita literatura / Bibliography
  % ----------------------------------------------
\ifslovak
  \makeatletter
  \def\@openbib@code{\addcontentsline{toc}{chapter}{Literatúra}}
  \makeatother
  \bibliographystyle{bib-styles/slovakiso}
\else
  \ifczech
    \makeatletter
    \def\@openbib@code{\addcontentsline{toc}{chapter}{Literatura}}
    \makeatother
    \bibliographystyle{bib-styles/czechiso}
  \else 
    \makeatletter
    \def\@openbib@code{\addcontentsline{toc}{chapter}{Bibliography}}
    \makeatother
    \bibliographystyle{bib-styles/englishiso}
  %  \bibliographystyle{alpha}
  \fi
\fi
  \begin{flushleft}
  \bibliography{projekt-20-literatura-bibliography}
  \end{flushleft}

  % vynechani stranky v oboustrannem rezimu
  % Skip the page in the two-sided mode
  \iftwoside
    \cleardoublepage
  \fi

  % Prilohy / Appendices
  % ---------------------------------------------
  \appendix
\ifczech
  \renewcommand{\appendixpagename}{Přílohy}
  \renewcommand{\appendixtocname}{Přílohy}
  \renewcommand{\appendixname}{Příloha}
\fi
\ifslovak
  \renewcommand{\appendixpagename}{Prílohy}
  \renewcommand{\appendixtocname}{Prílohy}
  \renewcommand{\appendixname}{Príloha}
\fi
%  \appendixpage

% vynechani stranky v oboustrannem rezimu
% Skip the page in the two-sided mode
%\iftwoside
%  \cleardoublepage
%\fi
  
\ifslovak
%  \section*{Zoznam príloh}
%  \addcontentsline{toc}{section}{Zoznam príloh}
\else
  \ifczech
%    \section*{Seznam příloh}
%    \addcontentsline{toc}{section}{Seznam příloh}
  \else
%    \section*{List of Appendices}
%    \addcontentsline{toc}{section}{List of Appendices}
  \fi
\fi
  \startcontents[chapters]
  \setlength{\parskip}{0pt}
  % seznam příloh / list of appendices
  % \printcontents[chapters]{l}{0}{\setcounter{tocdepth}{2}}
  
  \ifODSAZ
    \setlength{\parskip}{0.5\bigskipamount}
  \else
    \setlength{\parskip}{0pt}
  \fi
  
  % vynechani stranky v oboustrannem rezimu
  \iftwoside
    \cleardoublepage
  \fi
  
  % Přílohy / Appendices
  % Tento soubor nahraďte vlastním souborem s přílohami (nadpisy níže jsou pouze pro příklad)
% This file should be replaced with your file with an appendices (headings below are examples only)

% Umístění obsahu paměťového média do příloh je vhodné konzultovat s vedoucím
% Placing of table of contents of the memory media here should be consulted with a supervisor
%\chapter{Obsah přiloženého paměťového média}

%\chapter{Manuál}

%\chapter{Konfigurační soubor} % Configuration file

%\chapter{RelaxNG Schéma konfiguračního souboru} % Scheme of RelaxNG configuration file

%\chapter{Plakát} % poster

\chapter{Jak pracovat s touto šablonou}
\label{jak}

V této příloze je uveden popis jednotlivých částí šablony, po kterém následuje stručný návod, jak s touto šablonou pracovat. Pokud po jejím přečtení k šabloně budete mít nějaké dotazy, připomínky apod., neváhejte a napište na e-mail sablona@fit.vutbr.cz.

\section*{Popis částí šablony}

Po rozbalení šablony naleznete následující soubory a adresáře:
\begin{DESCRIPTION}
  \item [bib-styles] Styly literatury (viz níže). 
  \item [obrazky-figures] Adresář pro Vaše obrázky. Nyní obsahuje placeholder.pdf (tzv. TODO obrázek, který lze použít jako pomůcku při tvorbě technické zprávy), který se s prací neodevzdává. Název adresáře je vhodné zkrátit, aby byl jen ve zvoleném jazyce.
  \item [template-fig] Obrázky šablony (znak VUT).
  \item [fitthesis.cls] Šablona (definice vzhledu).
  \item [Makefile] Makefile pro překlad, počítání normostran, sbalení apod. (viz níže).
  \item [projekt-01-kapitoly-chapters.tex] Soubor pro Váš text (obsah nahraďte).
  \item [projekt-20-literatura-bibliography.bib] Seznam literatury (viz níže).
  \item [projekt-30-prilohy-appendices.tex] Soubor pro přílohy (obsah nahraďte).
  \item [projekt.tex] Hlavní soubor práce -- definice formálních částí.
\end{DESCRIPTION}

Výchozí styl literatury (czechiso) je od Ing. Martínka, přičemž slovenská a anglická verze (slovakiso a englishiso) jsou jeho překlady s drobnými modifikacemi. Oproti normě jsou v~něm určité odlišnosti, ale na FIT je dlouhodobě akceptován. Alternativně můžete využít styl od Ing. Radima Loskota nebo od Ing. Radka Pyšného\footnote{BP Ing. Radka Pyšného \url{http://www.fit.vutbr.cz/study/DP/BP.php?id=7848}}. Alternativní styly obsahují určitá vylepšení, ale zatím nebyly řádně otestovány větším množstvím uživatelů. Lze je považovat za beta verze pro zájemce, kteří svoji práci chtějí mít dokonalou do detailů a neváhají si nastudovat detaily správného formátování citací, aby si mohli ověřit, že je vysázený výsledek v pořádku.

\begin{samepage}
Makefile kromě překladu do PDF nabízí i další funkce:
\begin{itemize}
  \item přejmenování souborů (viz níže),
  \item počítání normostran,
  \item spuštění vlny pro doplnění nezlomitelných mezer,
  \item sbalení výsledku pro odeslání vedoucímu ke kontrole (zkontrolujte, zda sbalí všechny Vámi přidané soubory, a případně doplňte).
\end{itemize}
\end{samepage}

Nezapomeňte, že vlna neřeší všechny nezlomitelné mezery. Vždy je třeba manuální kontrola, zda na konci řádku nezůstalo něco nevhodného -- viz Internetová jazyková příručka\footnote{Internetová jazyková příručka \url{http://prirucka.ujc.cas.cz/?id=880}}.

\paragraph {Pozor na číslování stránek!} Pokud má obsah 2 strany a na 2. jsou jen \uv{Přílohy} a~\uv{Seznam příloh} (ale žádná příloha tam není), z nějakého důvodu se posune číslování stránek o 1 (obsah \uv{nesedí}). Stejný efekt má, když je na 2. či 3. stránce obsahu jen \uv{Literatura} a~je možné, že tohoto problému lze dosáhnout i jinak. Řešení je několik (od~úpravy obsahu, přes nastavení počítadla až po sofistikovanější metody). \textbf{Před odevzdáním proto vždy překontrolujte číslování stran!}


\section*{Doporučený postup práce se šablonou}

\begin{enumerate}
  \item \textbf{Zkontrolujte, zda máte aktuální verzi šablony.} Máte-li šablonu z předchozího roku, na stránkách fakulty již může být novější verze šablony s~aktualizovanými informacemi, opravenými chybami apod.
  \item \textbf{Zvolte si jazyk}, ve kterém budete psát svoji technickou zprávu (česky, slovensky nebo anglicky) a svoji volbu konzultujte s vedoucím práce (nebyla-li dohodnuta předem). Pokud Vámi zvoleným jazykem technické zprávy není čeština, nastavte příslušný parametr šablony v souboru projekt.tex (např.: \verb|documentclass[english]{fitthesis}| a přeložte prohlášení a poděkování do~angličtiny či slovenštiny.
  \item \textbf{Přejmenujte soubory.} Po rozbalení je v šabloně soubor \texttt{projekt.tex}. Pokud jej přeložíte, vznikne PDF s technickou zprávou pojmenované \texttt{projekt.pdf}. Když vedoucímu více studentů pošle \texttt{projekt.pdf} ke kontrole, musí je pracně přejmenovávat. Proto je vždy vhodné tento soubor přejmenovat tak, aby obsahoval Váš login a (případně zkrácené) téma práce. Vyhněte se však použití mezer, diakritiky a speciálních znaků. Vhodný název může být např.: \uv{\texttt{xlogin00-Cisteni-a-extrakce-textu.tex}}. K přejmenování můžete využít i přiložený Makefile:
\begin{verbatim}
make rename NAME=xlogin00-Cisteni-a-extrakce-textu
\end{verbatim}
  \item Vyplňte požadované položky v souboru, který byl původně pojmenován \texttt{projekt.tex}, tedy typ, rok (odevzdání), název práce, svoje jméno, ústav (dle zadání), tituly a~jméno vedoucího, abstrakt, klíčová slova a další formální náležitosti.
  \item Nahraďte obsah souborů s kapitolami práce, literaturou a přílohami obsahem svojí technické zprávy. Jednotlivé přílohy či kapitoly práce může být výhodné uložit do~samostatných souborů -- rozhodnete-li se pro toto řešení, je doporučeno zachovat konvenci pro názvy souborů, přičemž za číslem bude následovat název kapitoly. 
  \item Nepotřebujete-li přílohy, zakomentujte příslušnou část v \texttt{projekt.tex} a příslušný soubor vyprázdněte či smažte. Nesnažte se prosím vymyslet nějakou neúčelnou přílohu jen proto, aby daný soubor bylo čím naplnit. Vhodnou přílohou může být obsah přiloženého paměťového média.
  \item Zadání, které si stáhnete v PDF z IS FIT (odkaz \uv{Zadání pro vložení do práce} či \uv{Thesis assignment}), uložte do souboru \texttt{zadani.pdf} a povolte jeho vložení do práce parametrem šablony v \texttt{projekt.tex} (\verb|documentclass[zadani]{fitthesis}|).
  \item Nechcete-li odkazy tisknout barevně (tedy červený obsah -- bez konzultace s vedoucím nedoporučuji), budete pro tisk vytvářet druhé PDF s tím, že nastavíte parametr šablony pro tisk: (\verb|documentclass[zadani,print]{fitthesis}|). Budete-li tisknout barevně, místo \texttt{print} použijte parametr \texttt{cprint}. Barevné logo se nesmí tisknout černobíle!
  \item Vzor desek, do kterých bude práce vyvázána, si vygenerujte v informačním systému fakulty u zadání. Pro disertační práci lze zapnout parametrem v šabloně \texttt{cover} (více naleznete v souboru \texttt{fitthesis.cls}).
  \item Nezapomeňte, že zdrojové soubory i (obě verze) PDF musíte odevzdat na CD či jiném médiu přiloženém k technické zprávě.
\end{enumerate}

Obsah práce se generuje standardním příkazem \tt \textbackslash tableofcontents \rm (zahrnut v šabloně). Přílohy jsou v něm uvedeny úmyslně.

\subsection*{Pokyny pro oboustranný tisk}
\begin{itemize}
\item \textbf{Oboustranný tisk je doporučeno konzultovat s vedoucím práce.}
\item Je-li práce tištěna oboustranně a její tloušťka je menší než tloušťka desek, nevypadá to dobře.
\item Zapíná se parametrem šablony: \verb|\documentclass[twoside]{fitthesis}|
\item Po vytištění oboustranného listu zkontrolujte, zda je při prosvícení sazební obrazec na obou stranách na stejné pozici. Méně kvalitní tiskárny s duplexní jednotkou mají často posun o 1--3 mm. Toto může být u některých tiskáren řešitelné tak, že vytisknete nejprve liché stránky, pak je dáte do stejného zásobníku a vytisknete sudé.
\item Za titulním listem, obsahem, literaturou, úvodním listem příloh, seznamem příloh a případnými dalšími seznamy je třeba nechat volnou stránku, aby následující část začínala na liché stránce (\textbackslash cleardoublepage).
\item  Konečný výsledek je nutné pečlivě překontrolovat.
\end{itemize}

\subsection*{Styl odstavců}

Odstavce se zarovnávají do bloku a pro jejich formátování existuje více metod. U papírové literatury je častá metoda s~použitím odstavcové zarážky, kdy se u~jednotlivých odstavců textu odsazuje první řádek odstavce asi o~jeden až dva čtverčíky (vždy o~stejnou, předem zvolenou hodnotu), tedy přibližně o~dvě šířky velkého písmene M základního textu. Poslední řádek předchozího odstavce a~první řádek následujícího odstavce se v~takovém případě neoddělují svislou mezerou. Proklad mezi těmito řádky je stejný jako proklad mezi řádky uvnitř odstavce. \cite{fitWeb} 

Další metodou je odsazení odstavců, které je časté u elektronické sazby textů. První řádek odstavce se při této metodě neodsazuje a mezi odstavce se vkládá vertikální mezera o~velikosti 1/2 řádku. Obě metody lze v kvalifikační práci použít, nicméně často je vhodnější druhá z uvedených metod. Metody není vhodné kombinovat.

Jeden z výše uvedených způsobů je v šabloně nastaven jako výchozí, druhý můžete zvolit parametrem šablony \uv{\tt odsaz\rm }.

\subsection*{Užitečné nástroje}
\label{nastroje}

Následující seznam není výčtem všech využitelných nástrojů. Máte-li vyzkoušený osvědčený nástroj, neváhejte jej využít. Pokud však nevíte, který nástroj si zvolit, můžete zvážit některý z následujících:

\begin{description}
	\item[\href{http://miktex.org/download}{MikTeX}] \LaTeX{} pro Windows -- distribuce s jednoduchou instalací a vynikající automatizací stahování balíčků. MikTex obsahuje i vlastní editor, ale spíše doporučuji TeXstudio.
	\item[\href{http://texstudio.sourceforge.net/}{TeXstudio}] Přenositelné opensource GUI pro \LaTeX{}.  Ctrl+klik umožňuje přepínat mezi zdrojovým textem a PDF. Má integrovanou kontrolu pravopisu\footnote{Českou kontrolu pravopisu lze doinstalovat z \url{https://extensions.openoffice.org/de/project/czech-dictionary-pack-ceske-slovniky-cs-cz}}, zvýraznění syntaxe apod. Pro jeho využití je nejprve potřeba nainstalovat MikTeX případně jinou \LaTeX ovou distribuci.
	\item[\href{http://www.winedt.com/}{WinEdt}] Ve Windows je dobrá kombinace WinEdt + MiKTeX. WinEdt je GUI pro Windows, pro jehož využití je nejprve potřeba nainstalovat \href{http://miktex.org/download}{MikTeX} či \href{http://www.tug.org/texlive/}{TeX Live}. 
	\item[\href{http://kile.sourceforge.net/}{Kile}] Editor pro desktopové prostředí KDE (Linux). Umožňuje živé zobrazení náhledu. Pro jeho využití je potřeba mít nainstalovaný \href{http://www.tug.org/texlive/}{TeX Live} a Okular. 
	\item[\href{http://jabref.sourceforge.net/download.php}{JabRef}] Pěkný a jednoduchý program v Javě pro správu souborů s bibliografií (literaturou). Není potřeba se nic učit -- poskytuje jednoduché okno a formulář pro editaci položek.
	\item[\href{https://inkscape.org/en/download/}{InkScape}] Přenositelný opensource editor vektorové grafiky (SVG i PDF). Vynikající nástroj pro tvorbu obrázků do odborného textu. Jeho ovládnutí je obtížnější, ale výsledky stojí za to.
	\item[\href{https://git-scm.com/}{GIT}] Vynikající pro týmovou spolupráci na projektech, ale může výrazně pomoci i jednomu autorovi. Umožňuje jednoduché verzování, zálohování a přenášení mezi více počítači.
	\item[\href{http://www.overleaf.com/}{Overleaf}] Online nástroj pro \LaTeX{}. Přímo zobrazuje náhled a umožňuje jednoduchou spolupráci (vedoucí může průběžně sledovat psaní práce), vyhledávání ve zdrojovém textu kliknutím do PDF, kontrolu pravopisu apod. Zdarma jej však lze využít pouze s určitými omezeními (někomu stačí na disertaci, jiný na ně může narazit i při psaní bakalářské práce) a pro dlouhé texty je pomalejší. Pro vedoucí má FIT licenci a~v~případě, že student narazí na omezení, je s pomocí vedoucího situace řešitelná.
\end{description}

Pozn.: Overleaf nepoužívá Makefile v šabloně -- aby překlad fungoval, je nutné kliknout pravým tlačítkem na \tt projekt.tex \rm a zvolit \uv{Set as Main File}.

\chapter{Psaní anglického textu}
\label{anglicky}
Tato příloha je převzata ze stránek doc. Černockého \cite{CernockyEnglish}.

Spousta lidí píše zprávy k projektům anglicky (a to je dobře!), ale dělá v nich spoustu zbytečných chyb (a to je špatně). Nejsem angličtinář, ale tento jazyk už nějakých pár let používám k psaní, čtení i komunikaci -- tato příloha obsahuje pár důležitých věcí. Pokud chcete napsat práci nebo článek opravdu 100\,\% dobře, nezbude Vám než si najmout rodilého mluvčího (a to by měl by být trochu technicky zdatný a aspoň trochu rozumět tomu, co píšete, ať to neskončí ještě hůř \ldots).

\section*{Obecně}

\begin{itemize}
  \item{Předtím, než budete sami něco psát, si přečtěte pár anglických technických článků a~zkuste si zapamatovat a získat \uv{obecný pocit}, jak se to píše.}
  \item{Používejte vždy korektor pravopisu -- zabudovaný ve Wordu, nebo v OpenOffice, pokud děláte na Linuxu, tak ISPELL a další (většina editorů pro \LaTeX{} má již kontrolu pravopisu integrovanou).}
  \item{Používejte korektor gramatiky. Nevím, jestli je nějaký dostupný na Linuxu, ale ten ve Wordu celkem slušně funguje a pokud Vám něco zelené podtrhne, je tam většinou opravdu chyba. Můžete do něj nakopírovat i zdrojový text pro \LaTeX{}, opravit, a pak uložit opět jako čistý text. Pokud používáte vim, je tam zabudovaný také a zvládne jak překlepy, tak základní gramatiku. V dokumentu \texttt{diplomka.tex} na první řádek napište: 
  \begin{verbatim}
    % vim:spelllang=en_us:spell
  \end{verbatim}
  (případně \texttt{en\_gb} pro OED angličtinu)
  \textit{Poznámka editora:} Existuje i velmi dobrý online nástroj Grammarly\footnote{\url{https://www.grammarly.com/}}, který je v základní verzi zdarma. 
  }
  \item{Online slovníky jsou dobré, ale nepoužívejte je slepě. Většinou dají více variant a ne každá je správně.}
  \item{\begin{samepage}Na vyhledávání a zjištění, co bude asi správné, můžete použít Google. Např.: nevíte, jak se řekne \uv{výhoda tohoto přístupu}. Slovník na seznam.cz dá asi 10 variant. Napište je postupně do vyhledávání na googlu:
  \begin{verbatim}
    "advantage of this approach" 1100000 hits
    "privilege of this approach" 6 hits
    "facility of this approach"  16 hits
  \end{verbatim}
  Neříkám, že je to 100\,\% správně, ale je to určité vodítko. Toto se dá použít i~na~dohledání správných spojek (třeba \uv{among two cases} nebo \uv{between two cases}?)\end{samepage}}
\end{itemize}
       
\section*{SVOMPT a shoda}

Struktura anglické věty je SVOPMT: SUBJECT VERB OBJECT MANNER PLACE TIME a přes to nejede vlak! Není volná jako v češtině. Jinak to je maximálně v nějaké divadelní hře, kde je potřeba něco zdůraznit. Hlavně podmět tam musí vždycky být, na to se často zapomíná, protože v CZ/SK může být zamlčený nebo nevyjádřený. SVOMPT platí i ve vedlejších větách!
\begin{verbatim}
  BAD: We have shown that is faster than the other function. 
  GOOD: We have shown that it is faster than the other function. 
\end{verbatim}

\noindent Shoda podmětu s přísudkem -- zní to šíleně, ale dělá se v tom spousta chyb. 

\begin{verbatim}
  he has 
  the users have 
  people were 
\end{verbatim}

\section*{Členy}

Členy v angličtině jsou noční můra a téměř nikdo z nás je nedává dobře. Základní pravidlo je, že když je něco určitého, musí předtím být \uv{the}. Členy musí být určitě u těchto spojení:
\begin{verbatim}
  the first, the second, ...
  the last
  the most (třetí stupeň přídavných jmen a príslovcí) ...
  the whole 
  the following 
  the figure, the table. 
  the left, the right - on the left pannel, from the left to the right ... 
\end{verbatim}

\noindent Naopak člen NESMÍ být, pokud používáte přesné označení obrázku, kapitoly, atd.
\begin{verbatim}
  in Figure 3.2
  in Chapter 7
  in Table 6.4
\end{verbatim}

\begin{samepage}
\noindent Pozor na \uv{a} vs. \uv{an}, řídí se to podle výslovnosti a ne podle toho, jak je slovo napsané, takže:
\begin{verbatim}
  an HMM
  an XML
  a universal model
  a user
\end{verbatim}
\end{samepage}

\section*{Slovesa}

Pozor na trpné tvary sloves -- u pravidelných je to většinou bez problémů, u nepravidelných často špatně, typicky
\begin{verbatim}
  packet was sent (ne send)
  approach was chosen (ne choosed)
\end{verbatim}
\noindent \ldots vetšinou to opraví korektor pravopisu, ale někdy ne. 

Pozor na časy, občas je v nich pěkný nepořádek. Pokud něco nějak obecně je, přítomný čas. Pokud jste něco udělali, minulý. Pokud to dalo nějaký výsledek a ten výsledek teď existuje a třeba ho nějak diskutujete, přítomný. Nepoužívejte příliš složité časy jako je předpřítomný a vůbec ne předminulý pokud nevíte přesně, co děláte.
\begin{verbatim}
  JFA is a technique that works for everyone in speaker recognition. 
  We implemented it according to Kenny's recipe in \cite{Kenny}. 
  12000 segments from NIST SRE 2006 were processed. When compared 
  with a GMM baseline, the results are completely bad. 
\end{verbatim}

\section*{Délka vět a struktura}

\begin{itemize}
  \item{Pište kratší věty a souvětí, pokud máte něco na 5 řádku, většinou se to nedá číst.}
  \item{Strukturujte věty pomocí čárek (více než v češtině!), hlavně po úvodu věty, po kterém začíná vlastní věta. Někdy se dává čárka i před \uv{and} (na rozdíl od češtiny)}
\end{itemize}
\begin{verbatim}
  In this chapter, we will investigate ... 
  The first technique did not work, the second did not work as well, 
  and the third one also did not work. 
\end{verbatim}

\section*{Specifika technického textu}

Píšete technicky text, proto nepoužívejte zkratky
\begin{verbatim}
  he's
  gonna
  Petr's working on ...
\end{verbatim}
\noindent a podobně. Jediné, které je tolerované, je \uv{doesn't}, ale neuděláte chybu, když napíšete \uv{does not}. 

\begin{samepage}
\noindent V technických textech se spíš používá trpný rod než činný: 
\begin{verbatim}
  BAD: In this chapter, I describe used programming languages. 
  GOOD: In this chapter, used programming languages are described.
\end{verbatim}
\end{samepage}

Pokud už činný použijete, dává se v technických textech spíše \uv{we}, i když na práci děláte sami. \uv{I}, \uv{my}, atd. se používají pouze tam, kde jde o to zdůraznit, že jde o Vaši osobu, tedy třeba v závěru nebo v popisu \uv{originál claims} v disertaci.

\paragraph{Časté chyby ve slovech}

\begin{itemize}
  \item{Pozor na jeho/její, není to it's, ale its }
  \item{Obrázek není picture, ale figure. }
  \item{Spojka \uv{než} je \uv{than}, ne \uv{then} -- bigger than this, smaller than this \ldots hrozně častá chyba! \uv{Then} je pak, potom.}
\end{itemize}


\chapter{Checklist} 
\label{checklist}
Tento checklist byl převzat ze šablony pro kvalifikační práce, která je k dispozici na blogu prof. Herouta \cite{Herout}, který s laskavým dovolením využil nápadu dr. Szökeho%
\footnote{\url{http://blog.igor.szoke.cz/2017/04/predstartovni-priprava-letu-neni.html}}. 

Velká bezpečnost letecké dopravy stojí z části na tom, že lidé kolem letadel mají \textbf{checklisty} na úplně každý, třeba rutinní a dobře zažitý, postup. Jako pilot strpí to, že bude trochu za blbce a opravdu tužtičkou do seznamu úkonů odškrtá dokonale zvládnuté akce, vytiskněte si a odškrtejte před odevzdáním diplomky i vy tento checklist a vyhněte se tak častým chybám, které by mohly mít až fatální následky na výsledné hodnocení Vaší práce.

\subsubsection*{Struktura}
\begin{checklist}
	\item Už ze samotných názvů a struktury kapitol je patrné, že bylo splněno zadání.
	\item V textu se nevyskytuje kapitola, která by měla méně než čtyři strany (kromě úvodu a závěru). Pokud ano, radil(a) jsem se o tom s vedoucím a ten to schválil.
\end{checklist}

\subsubsection*{Obrázky a grafy}
\begin{checklist}
	\item Všechny obrázky a tabulky byly zkontrolovány a jsou poblíž místa, odkud jsou z textu odkazovány, takže nebude problém je najít.
	\item Všechny obrázky a tabulky mají takový popisek, že celý obrázek dává smysl sám o~sobě, bez čtení dalšího textu. Vůbec nevadí, když má popisek několik řádků.
	\item Pokud je obrázek převzatý, tak je to v popisku zmíněno: \uv{Převzato z [X].}
	\item Písmenka ve všech obrázcích používají font podobné velikosti, jako je okolní text (ani výrazně větší, ani výrazně menší).
	\item Grafy a schémata jsou vektorově (tj. v PDF).
	\item Snímky obrazovky nepoužívají ztrátovou kompresi (jsou v PNG).
	\item Všechny obrázky jsou odkázány z textu.
	\item Grafy mají popsané osy (název osy, jednotky, hodnoty) a podle potřeby mřížku.
\end{checklist}

\subsubsection*{Rovnice}
\begin{checklist}
	\item Identifikátory a jejich indexy v rovnicích jsou jednopísmenné (kromě nečastých zvláštních případů jako $t_\mathrm{max}$).
	\item Rovnice jsou číslovány.
	\item Za (nebo vzácně před) rovnicí jsou vysvětleny všechny proměnné a funkce, které zatím vysvětleny nebyly.
\end{checklist}

\subsubsection*{Citace}
\begin{checklist}
    \item \textbf{Všechny použité zdroje jsou citovány.}
	\item Adresy URL odkazující na služby, projekty, zdroje, github apod. jsou odkazovány pomocí \verb|\footnote{\url{...}}|.
    \item Všechny citace používají správné typy.
	\item Citace mají autora, název, vydavatele (název konference), rok vydání.  Když některá nemá, je to dobře zdůvodněný zvláštní případ a vedoucí to odsouhlasil.
\end{checklist}

\subsubsection*{Typografie}
\begin{checklist}
	\item Žádný řádek nepřetéká přes pravý okraj.
	\item Na konci řádku nikde není jednopísmenná předložka (spraví to nedělitelná mezera $\sim$).
	\item Číslo obrázku, tabulky, rovnice, citace není nikde první na novém řádku (spraví to nedělitelná mezera $\sim$).
	\item Před číselným odkazem na poznámku pod čarou nikde není mezera (to jest vždy takto\footnote{příklad poznámky pod čarou}, nikoliv takto \footnote{jiný příklad poznámky pod čarou}).
\end{checklist}

\subsubsection*{Jazyk}
\begin{checklist}
    \item Použil jsem kontrolu pravopisu a v textu nikde nejsou překlepy.
	\item Nechal jsem si text přečíst od (alespoň) jednoho dalšího člověka, který umí dobře česky / anglicky / slovensky.
	\item V práci psané česky nebo slovensky abstrakt zkontroloval někdo, kdo umí opravdu dobře anglicky.
	\item V textu se nikde nepoužívá druhá mluvnická osoba (vy/ty).
	\item Když se v textu vyskytuje první mluvnická osoba (já, my), vždy se popisuje subjektivní záležitost (\textit{rozhodl jsem se}, \textit{navrhl jsem}, \textit{zaměřil jsem se na}, \textit{zjistil jsem} apod.).
	\item V textu se nikde nepoužívají hovorové výrazy.
	\item V českém či slovenském textu se zbytečně nepoužívají anglické výrazy, které mají ustálené české překlady. Např. slovo \textit{defaultní} se nahradí např. slovem \textit{implicitní} nebo \textit{výchozí}.
\end{checklist}

\subsubsection*{Výsledek na datovém médiu, tj. software}
\begin{checklist}
	\item Mám připravené nepřepisovatelné datové médium 
      \begin{itemize}
	  		\item CD-R,
            \item DVD-R,
            \item DVD+R ve formátu ISO9660 (s rozšířením RockRidge a/nebo Jolliet) nebo UDF,
            \item paměťová karta SD (Secure Digital) ve formátu FAT32 nebo exFAT s nastavenou ochranou proti přepisu.
      \end{itemize}
	\item Pokud je výsledek online (služba, aplikace, \dots), URL je viditelně v úvodu a závěru, aby bylo jasné, kde výsledek hledat.
	\item Na médiu nechybí povinné: 
    	\begin{itemize}
    		\item zdrojové kódy (např. Matlab, C/C++,Python, \dots)
            \item knihovny potřebné pro překlad,
            \item přeložené řešení,
            \item PDF s technickou zprávou (je-li pro tisk 2. verze, tak obě),
            \item zdrojový kód zprávy (\LaTeX), 
    	\end{itemize}
        a případně volitelně po dohodě s vedoucím práce
		\begin{itemize}
			\item relevantní (např. testovací) data, 
            \item demonstrační video,
            \item PDF plakátku,
            \item \dots
		\end{itemize}        
	\item Zdrojové kódy jsou refaktorovány, komentovány a označeny hlavičkou s autorstvím, takže se v nich snadno vyzná i někdo další, než sám autor.
    \item Jakákoliv převzatá část zdrojového kódu je řádně citována -- tedy označena úvodním a v případě převzetí více řádků i ukončovacím komentářem. Komentář obsahuje vše, co vyžaduje licence uvedená na webu (vždy je nutné se ji pokusit najít -- např. Stack Overflow\footnote{\url{https://stackoverflow.blog/2009/06/25/attribution-required/}} má striktní pravidla pro citace).
\end{checklist}

\subsubsection*{Odevzdání}

\begin{checklist}
\item Chci práci (na max. 3 roky) utajit? Pokud ano, nejpozději měsíc před termínem odevzdání práce si podám žádost (v IS), ke které přiložím případné stanovisko firmy, jejíž duševní vlastnictví je třeba chránit.
\item Mám splněný minimální počet normostran textu (lze spočítat pomocí Makefile a~odhadem přičíst obrázky). Pokud jsem těsně pod minimem, konzultoval(a) jsem to s~vedoucím.
\item Pokud chci tisknout oboustranně, konzultoval(a) jsem to s~vedoucím a mám správně nastavenou šablonu. Kapitoly začínají na liché stránce.
\item Technickou zprávu mám v deskách z knihařství (min. 1 výtisk, při utajení oba).
\item Za titulním listem práce je zadání (tzn. mám jej stažené z IS a vložené do šablony).
\item V IS jsou abstrakty a klíčová slova.
\item V IS je PDF práce (s klikatelnými odkazy).
\item Oba výtisky práce jsou podepsané.
\item V jednom (při utajení obou) výtisku práce je paměťové médium, na kterém je fixkou napsaný login (fixku na CD lze zapůjčit v knihovně, na Studijním oddělení nebo až při odevzdání).
\end{checklist}


\chapter{\LaTeX pro začátečníky}
\label{latex}

V této kapitole jsou uvedeny některé často využívané balíčky a příkazy pro \LaTeX{}, které mohou být při tvorbě práce potřeba.

\subsection*{Užitečné balíčky}

Studenti při sazbě textu často řeší stejné problémy. Některé z nich lze vyřešit následujícími balíčky pro \LaTeX:

\begin{itemize}
  \item \verb|amsmath| -- rozšířené možnosti sazby rovnic,
  \item \verb|float, afterpage, placeins| -- úprava umístění obrázků/tabulek (specifikátor \texttt{H}),
  \item \verb|fancyvrb, alltt| -- úpravy vlastností prostředí Verbatim, 
  \item \verb|makecell| -- rozšíření možností tabulek,
  \item \verb|pdflscape, rotating| -- natočení stránky o 90 stupňů (pro obrázek či tabulku),
  \item \verb|hyphenat| -- úpravy dělení slov,
  \item \verb|picture, epic, eepic| -- přímé kreslení obrázků.
\end{itemize}

Některé balíčky jsou využity přímo v šabloně (v dolní části souboru \texttt{fitthesis.cls}). Nahlédnutí do jejich dokumentace může být rovněž velmi užitečné.

Sloupec tabulky zarovnaný vlevo s pevnou šířkou je v šabloně definovaný \uv{L} (používá se jako \uv{p}).

Pro odkazování v rámci textu použijte příkaz \verb|\ref{navesti}|. Podle umístění návěští se bude jednat o~číslo kapitoly, podkapitoly, obrázku, tabulky nebo podobného číslovaného prvku). Pokud chcete odkázat stránku práce, použijte příkaz \verb|pageref{navesti}|. Pro citaci literárního odkazu \verb|\cite{identifikator}|. Pro odkazy na rovnice lze použít příkaz \verb|\eqref{navesti}|.

Znak \,--\, (pomlčka) se V \LaTeX u vkládá jako dvě mínus za sebou: -{}-.

\subsection*{Často využívané příkazy pro \LaTeX{}}
\label{sec:Fragments}

Doporučuji nahlédnout do zdrojového textu této podkapitoly a podívat se, jak jsou následující ukázky vysázeny. Ve zdrojovém textu jsou i pomocné komentáře.

% Sloupec zarovnaný vlevo s pevnou šířkou je v šabloně definovaný "L" (používá se jako p)

Příklad tabulky:
\begin{table}[H]
	\vskip6pt
	\caption{Tabulka hodnocení} 
    \vskip6pt
	\centering
	\begin{tabular}{llr}
		\toprule
		\multicolumn{2}{c}{Jméno} \\
		\cmidrule(r){1-2}
		Jméno & Příjmení & Hodnocení \\
		\midrule
		Jan & Novák & $7.5$ \\
		Petr & Novák & $2$ \\
		\bottomrule
	\end{tabular}
	\label{tab:ExampleTable}
\end{table}

% Ohraničení lze upravit dle potřeby:
% http://latex-community.org/forum/viewtopic.php?f=45&t=24323
% http://tex.stackexchange.com/questions/58163/problem-with-multirow-and-table-cell-borders
% http://tex.stackexchange.com/questions/79369/formatting-table-border-and-text-alignment-in-latex-table

\noindent Příklad rovnice:
\begin{equation}
	\cos^3 \theta =\frac{1}{4}\cos\theta+\frac{3}{4}\cos 3\theta
	\label{eq:rovnice2}
\end{equation}
a dvou horizontálně zarovnaných rovnic: % znak & řídí zarovnání
\begin{align} 
    \label{eq:soustava}
	3x &= 6y + 12 \\
	x &= 2y + 4 
\end{align}

Pokud je třeba rovnici citovat v textu, lze použít příkaz \texttt{\\eqref}. Například na rovnici výše lze odkázat~\eqref{eq:rovnice2}. Pokud chcete srovnat číslo rovnic u soustavy, lze použít prostředí \texttt{split}:
\begin{equation} \label{eq:soustavaSrovnana}
\begin{split}
	3x &= 6y + 12 \\
	x &= 2y + 4
\end{split}
\end{equation}

Matematické symboly ($\alpha$) a výrazy lze umístit i do textu $\cos\pi=-1$ a mohou být i~v~poznámce pod čarou%
\footnote{Vzorec v poznámce pod čarou: $\cos\pi=-1$}.

Obrázek~\ref{sirokyObrazek} ukazuje široký obrázek složený z více menších obrázků. Klasický rastrový obrázek se vkládá tak, jak je vidět na obrázku \ref{keepCalm}.

% Využití \begin{figure*} způsobí, že obrázek zabere celou šířku stránky. Takový obrázek dříve mohl být pouze na začátku stránky, případně na konci s využitím balíčku dblfloatfix (případné [h] se ignorovalo a [H] obrázek odstraní). Nové verze LaTeXu už umí i [h].
\begin{figure*}[h]\centering
  \centering
  \includegraphics[width=\linewidth,height=1.7in]{obrazky-figures/placeholder.pdf}\\[1pt]
  \includegraphics[width=0.24\linewidth]{obrazky-figures/placeholder.pdf}\hfill
  \includegraphics[width=0.24\linewidth]{obrazky-figures/placeholder.pdf}\hfill
  \includegraphics[width=0.24\linewidth]{obrazky-figures/placeholder.pdf}\hfill
  \includegraphics[width=0.24\linewidth]{obrazky-figures/placeholder.pdf}
  \caption{\textbf{Široký obrázek.} Obrázek může být složen z více menších obrázků. Chcete-li se na tyto dílčí obrázky odkazovat z textu, využijte balíček \texttt{subcaption}.}
  \label{sirokyObrazek}
\end{figure*}

\begin{figure}[hbt]
	\centering
	\includegraphics[width=0.3\textwidth]{obrazky-figures/keep-calm.png}
	\caption{Dobrý text je špatným textem, který byl několikrát přepsán. Nebojte se prostě něčím začít.}
	\label{keepCalm}
\end{figure}

Další často využívané příkazy naleznete ve zdrojovém textu ukázkového obsahu této šablony.


  
  % Kompilace po částech (viz výše, nutno odkomentovat)
  % Compilation piecewise (see above, it is necessary to uncomment it)
  %\subfile{projekt-30-prilohy-appendices}
  
\end{document}
